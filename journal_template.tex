\documentclass[
  man,
  longtable,
  nolmodern,
  notxfonts,
  notimes,
  colorlinks=true,linkcolor=blue,citecolor=blue,urlcolor=blue]{apa7}

\usepackage{amsmath}
\usepackage{amssymb}




\RequirePackage{longtable}
\RequirePackage{threeparttablex}

\makeatletter
\renewcommand{\paragraph}{\@startsection{paragraph}{4}{\parindent}%
	{0\baselineskip \@plus 0.2ex \@minus 0.2ex}%
	{-.5em}%
	{\normalfont\normalsize\bfseries\typesectitle}}

\renewcommand{\subparagraph}[1]{\@startsection{subparagraph}{5}{0.5em}%
	{0\baselineskip \@plus 0.2ex \@minus 0.2ex}%
	{-\z@\relax}%
	{\normalfont\normalsize\bfseries\itshape\hspace{\parindent}{#1}\textit{\addperi}}{\relax}}
\makeatother




\usepackage{longtable, booktabs, multirow, multicol, colortbl, hhline, caption, array, float, xpatch}
\setcounter{topnumber}{2}
\setcounter{bottomnumber}{2}
\setcounter{totalnumber}{4}
\renewcommand{\topfraction}{0.85}
\renewcommand{\bottomfraction}{0.85}
\renewcommand{\textfraction}{0.15}
\renewcommand{\floatpagefraction}{0.7}

\usepackage{tcolorbox}
\tcbuselibrary{listings,theorems, breakable, skins}
\usepackage{fontawesome5}

\definecolor{quarto-callout-color}{HTML}{909090}
\definecolor{quarto-callout-note-color}{HTML}{0758E5}
\definecolor{quarto-callout-important-color}{HTML}{CC1914}
\definecolor{quarto-callout-warning-color}{HTML}{EB9113}
\definecolor{quarto-callout-tip-color}{HTML}{00A047}
\definecolor{quarto-callout-caution-color}{HTML}{FC5300}
\definecolor{quarto-callout-color-frame}{HTML}{ACACAC}
\definecolor{quarto-callout-note-color-frame}{HTML}{4582EC}
\definecolor{quarto-callout-important-color-frame}{HTML}{D9534F}
\definecolor{quarto-callout-warning-color-frame}{HTML}{F0AD4E}
\definecolor{quarto-callout-tip-color-frame}{HTML}{02B875}
\definecolor{quarto-callout-caution-color-frame}{HTML}{FD7E14}

%\newlength\Oldarrayrulewidth
%\newlength\Oldtabcolsep


\usepackage{hyperref}




\providecommand{\tightlist}{%
  \setlength{\itemsep}{0pt}\setlength{\parskip}{0pt}}
\usepackage{longtable,booktabs,array}
\usepackage{calc} % for calculating minipage widths
% Correct order of tables after \paragraph or \subparagraph
\usepackage{etoolbox}
\makeatletter
\patchcmd\longtable{\par}{\if@noskipsec\mbox{}\fi\par}{}{}
\makeatother
% Allow footnotes in longtable head/foot
\IfFileExists{footnotehyper.sty}{\usepackage{footnotehyper}}{\usepackage{footnote}}
\makesavenoteenv{longtable}

\usepackage{graphicx}
\makeatletter
\newsavebox\pandoc@box
\newcommand*\pandocbounded[1]{% scales image to fit in text height/width
  \sbox\pandoc@box{#1}%
  \Gscale@div\@tempa{\textheight}{\dimexpr\ht\pandoc@box+\dp\pandoc@box\relax}%
  \Gscale@div\@tempb{\linewidth}{\wd\pandoc@box}%
  \ifdim\@tempb\p@<\@tempa\p@\let\@tempa\@tempb\fi% select the smaller of both
  \ifdim\@tempa\p@<\p@\scalebox{\@tempa}{\usebox\pandoc@box}%
  \else\usebox{\pandoc@box}%
  \fi%
}
% Set default figure placement to htbp
\def\fps@figure{htbp}
\makeatother







\usepackage{newtx}

\defaultfontfeatures{Scale=MatchLowercase}
\defaultfontfeatures[\rmfamily]{Ligatures=TeX,Scale=1}





\title{Training Current and Future Health-related Practitioners to
Accurately and Appropriately Disseminate Physical Activity Guidelines}


\shorttitle{Training Practitioners on Physical Activity Guidelines}


\usepackage{etoolbox}


\volume{13}







\authorsnames[{1},{2},{1},{1}]{Jafra D. Thomas,Winston Kennedy,Bethany
C. Lowe,Caroline N. Smith}







\authorsaffiliations{
{California Polytechnic State University},{Northeastern University}}




\leftheader{Thomas, Kennedy, Lowe and Smith}



\abstract{Current and future health-related practitioners have low
awareness of physical activity guidelines (PAGs) for general and
clinical populations. The purpose of the present study was to critically
appraise the quality of one 2021 draft training video, which was
designed to help current and future health related practitioners give
advice consistent with general adult PAGs. A descriptive qualitative
analysis was performed on open-ended responses provided by undergraduate
research assistants (or recent alumni) affiliated with the first
author's lab and uninvolved in the video's creation. Participation was
optional, anonymous, and through an online questionnaire, open for seven
days in April 2023 (14 invited, 8 participated, response rate =
57.14\%). Participant feedback was compared to applicable standards of
the RE-AIM framework (i.e., reach, efficacy, and adoption). Face
validity and other quality measures were determined through qualitative
analysis. The first author performed the descriptive analysis, and the
second author, acting as a critical friend, independently verified the
trustworthiness of the analysis. No issues were identified (i.e., a
succinct and verbatim analysis). Participants generally agreed the draft
video was clear, concise, informative, and interesting. Participants did
not perceive any major concerns with the video (e.g.,
non-offensive/biased), and their suggestions were used to finalize the
training video (e.g., to add closed captioning, further explain a
graph). Results confirmed the video had good face validity and could be
effective within real-world educational settings for current and future
health-related practitioners (e.g., low time burden, stimulating,
informative). Future research should investigate learning outcomes of
the video and its real-world implementation. }

\keywords{Exercise science, kinesiology, knowledge translation, health
communication, physical activity promotion guidelines, RE-AIM}

\authornote{\par{\addORCIDlink{Jafra D.
Thomas}{0000-0000-0000-0001}}\par{\addORCIDlink{Winston
Kennedy}{0000-0000-0000-0002}}\par{\addORCIDlink{Bethany C.
Lowe}{0000-0000-0000-0003}}\par{\addORCIDlink{Caroline N.
Smith}{0000-0000-0000-0004}} 
\par{ }
\par{   The authors have no conflicts of interest to disclose.    }
\par{Correspondence concerning this article should be addressed to Jafra
D. Thomas, Email: jafra.thomas@example.org}
}

\makeatletter
\let\endoldlt\endlongtable
\def\endlongtable{
\hline
\endoldlt
}
\makeatother
\RequirePackage{longtable}
\DeclareDelayedFloatFlavor{longtable}{table}

\urlstyle{same}



\makeatletter
\@ifpackageloaded{caption}{}{\usepackage{caption}}
\AtBeginDocument{%
\ifdefined\contentsname
  \renewcommand*\contentsname{Table of contents}
\else
  \newcommand\contentsname{Table of contents}
\fi
\ifdefined\listfigurename
  \renewcommand*\listfigurename{List of Figures}
\else
  \newcommand\listfigurename{List of Figures}
\fi
\ifdefined\listtablename
  \renewcommand*\listtablename{List of Tables}
\else
  \newcommand\listtablename{List of Tables}
\fi
\ifdefined\figurename
  \renewcommand*\figurename{Figure}
\else
  \newcommand\figurename{Figure}
\fi
\ifdefined\tablename
  \renewcommand*\tablename{Table}
\else
  \newcommand\tablename{Table}
\fi
}
\@ifpackageloaded{float}{}{\usepackage{float}}
\floatstyle{ruled}
\@ifundefined{c@chapter}{\newfloat{codelisting}{h}{lop}}{\newfloat{codelisting}{h}{lop}[chapter]}
\floatname{codelisting}{Listing}
\newcommand*\listoflistings{\listof{codelisting}{List of Listings}}
\makeatother
\makeatletter
\makeatother
\makeatletter
\@ifpackageloaded{caption}{}{\usepackage{caption}}
\@ifpackageloaded{subcaption}{}{\usepackage{subcaption}}
\makeatother

% From https://tex.stackexchange.com/a/645996/211326
%%% apa7 doesn't want to add appendix section titles in the toc
%%% let's make it do it
\makeatletter
\xpatchcmd{\appendix}
  {\par}
  {\addcontentsline{toc}{section}{\@currentlabelname}\par}
  {}{}
\makeatother

%% Disable longtable counter
%% https://tex.stackexchange.com/a/248395/211326

\usepackage{etoolbox}

\makeatletter
\patchcmd{\LT@caption}
  {\bgroup}
  {\bgroup\global\LTpatch@captiontrue}
  {}{}
\patchcmd{\longtable}
  {\par}
  {\par\global\LTpatch@captionfalse}
  {}{}
\apptocmd{\endlongtable}
  {\ifLTpatch@caption\else\addtocounter{table}{-1}\fi}
  {}{}
\newif\ifLTpatch@caption
\makeatother

\begin{document}

\maketitle


\setcounter{secnumdepth}{-\maxdimen} % remove section numbering

\setlength\LTleft{0pt}


\section{Introduction}\label{introduction}

While current and future health-related practitioners should be better
trained in using health behavior theory to counsel patients/clients and
to plan services (Thomas \& Cardinal, 2021), another looming issue
concerns their ability to accurately and appropriately disseminate
physical activity guidelines (PAGs, i.e., the frequency, intensity,
duration, and type of activities recommended for health-related fitness
and psychological well-being; U.S. Department of Health \& Human
Services, 2018; World Health Organization, 2020). Most health-related
practitioners and college majors may lack PAG-awareness or precise
knowledge of the guidelines (Cardinal et al., 2015; Vermeesch et al.,
2020; Zenko \& Ekkekakis, 2015). For example, Zenko and Ekkekakis (2015)
used an 11-item multiple-choice test to assess practical knowledge of
aerobic PAGs in a large and diverse sample of students and
professionals. The mean score for the number of correct answers was
42.87\% (95\% CI = 42.08-43.65\%). While job type and education level
were significant predictors of test scores, the overall low scores
suggest a need for improved training.

\section{Method}\label{method}

\subsection{Participants}\label{participants}

Participants were undergraduate research assistants or recent alumni
affiliated with the first author's lab and uninvolved in the video's
creation. Participation was optional, anonymous, and through an online
questionnaire, open for seven days in April 2023. Of the 14 individuals
invited, 8 participated (response rate = 57.14\%).

\subsection{Measures}\label{measures}

The primary measure was a qualitative questionnaire designed to gather
feedback on the draft training video. The questionnaire included
open-ended questions about the clarity, informativeness, and potential
effectiveness of the video. Responses were analyzed using a descriptive
qualitative approach and compared to applicable standards of the RE-AIM
framework (i.e., reach, efficacy, and adoption).

\subsection{Procedure}\label{procedure}

Participants were sent a link to view the draft training video and
complete the online questionnaire. The first author performed the
descriptive analysis of the responses, and the second author, acting as
a critical friend, independently verified the trustworthiness of the
analysis.

\section{Results}\label{results}

Participants generally agreed the draft video was clear, concise,
informative, and interesting. No issues were identified in the analysis
(i.e., a succinct and verbatim analysis). Participants did not perceive
any major concerns with the video (e.g., non-offensive/biased), and
their suggestions were used to finalize the training video (e.g., to add
closed captioning, further explain a graph).

\section{Discussion}\label{discussion}

Results confirmed the video had good face validity and could be
effective within real-world educational settings for current and future
health-related practitioners (e.g., low time burden, stimulating,
informative). The feedback provided valuable insights for finalizing the
training video and ensuring its effectiveness in disseminating physical
activity guidelines.

\subsection{Limitations and Future
Directions}\label{limitations-and-future-directions}

Limitations of this study include the small sample size and the fact
that participants were affiliated with the first author's lab. Future
research should investigate learning outcomes of the video and its
real-world implementation with a larger and more diverse sample of
health-related practitioners and students.

\subsection{Conclusion}\label{conclusion}

The training video developed in this study shows promise as an effective
tool for improving health-related practitioners' knowledge and
dissemination of physical activity guidelines. By addressing the current
gap in PAG awareness and knowledge, this video could contribute to
better health outcomes for patients and clients.






\end{document}
